\documentclass[11pt, a4paper]{report}

\usepackage[backend=bibtex, sorting=none]{biblatex}
\addbibresource{dossier_thematique_gd.bib}

\usepackage[T1]{fontenc}
\usepackage{arev, csquotes}
\usepackage[french]{babel}
\selectlanguage{french}

\usepackage{epigraph, calc}
\renewcommand{\epigraphsize}{\small}
\setlength{\epigraphwidth}{0.9\textwidth}
\renewcommand{\textflush}{flushright} \renewcommand{\sourceflush}{flushright}
\let\originalepigraph\epigraph 
\renewcommand\epigraph[2]{\originalepigraph{\textit{#1}}{\textsc{#2}}}

\usepackage{geometry}
\geometry{
    a4paper,
    left=18mm,
    right=18mm,
    bottom=15mm,
    top=20mm
}

\author{Jacques Soghomonyan}
\title{%
    L'Église arménienne, trace indélébile d'une civilisation mourante\bigbreak
    \large analyse de \textit{L'Église Arménienne apostolique image moderne et vivante de l'église primitive}\cite{livre:principale}
}
\date{08 mai 2023}

\pagenumbering{gobble}

% INTRO
%         - Accroche
%         - Présentation du livre + pourquoi ce choix (lien Grand Discours)
%         - Annonce de plan

% I_ De Quoi S'agit-Il ?
%         - Présentation de l'essai / roman + auteur
%         - Les critiques lors de la sortie (bonne presse ?)
%         - grands thèmes abordés dans le livre

% II_ Les Grands Thèmes De L'oeuvre
%         - Résumé de la thèse la plus simple à la plus complexe / forte

% III_ Analyse, Mise En Perspective, Critique
%         - Ce qu'on retient / qui nous interpelle / qu'on utilisera pour le Grand Discours
%         - 2 ou 3 citations ou chiffres+

% CONCLU
%          - Grand résumé général

\renewcommand{\thefootnote}{\textit{\alph{footnote}}}


\begin{document}
    \maketitle \pagebreak

    \section*{Un livre rare, témoin d'une culture riche}

    \epigraph{%
        De nos jours, un peuple est connu et jugé sur la valeur de sa culture.
        C'est par sa culture qu'un peuple devient une nation véritable et confirmée.
        En ce sens, le peuple arménien est sans aucun doute l'une des nations les plus confirmées du monde.}
    {Vazken $1^{er}$\cite{citation:catholicos,livre:principale}}

    Ce livre est un essai retraçant en partie l'histoire de l'Arménie et surtout de son Église. Il est assez court avec 117 pages et 11 chapitres.\\
    L'auteur, ici, analyse l'évolution de la culture arménienne et son interconnexion avec son Église,
    face aux tensions environnantes et aux multiples conquêtes de ses terres souveraines.\\
    Étant donné mon sujet de grand discours (\textit{Pour la reconnaissance de la culture arménienne face au génocide}),
    il me paraissait donc naturel de fouiller la bibliothèque familiale pour y trouver une piste. 
    J'y ai trouvé le livre\cite{livre:principale} étudié.
    C'est donc par quête de savoir vis-à-vis de la culture qui a engendré ma famille, que ce livre fait sens.
    \bigbreak
    
    Nous décomposerons l'analyse du livre avec : Une histoire arménienne riche, mais peu connu, puis, 
    Un peuple chrétien isolé de ses frères, enfin, 
    Un essai romancé qui témoigne de la fierté d'un peuple délaissé.

    \subsection*{Une histoire arménienne riche, mais peu connue}
    L'essai est principalement divisé en deux axes : 
    \begin{itemize}
        \item L'histoire du peuple arménien et son Église %
        \item L'analyse en profondeur de l'Église apostolique arménienne face au reste du \\ monde
    \end{itemize}
    \bigbreak

    Cet essai plonge dans l'histoire et relève les détailles et les spécificités de la conversion et le développement de l'Église apostolique en Arménie.\\
    En effet, cet essai relève plus d'une analyse historique que du récit analytique.
    Cependant, l'auteur vise large et a analysé de façon intégrale le sujet.
    \bigbreak

    Albert KHAZINEDJIAN né en 1937 à Blida (Algérie), et décédé en mars 2019 à Marseille a exercé le métier de médecin généraliste à Marseille.
    Né en Algérie, il suit ses parents dans leur immigration en France. 
    C'est en France qu'il servira l'Église catholique, c'est dans ce contexte qu'il découvre l'histoire et les divisions des écoles et Églises chrétiennes.

    \subsection*{Un peuple chrétien isolé de ses frères}

    Tout au long de l'essai, une idée prévaux, l'idée d'une Arménie seule en plein milieu du haut plateau mésopotamien. 
    Celle-ci même qui a fait rempart, tant de fois, face aux menaces extérieures sans aide, car non-chalcédonienne.\\
    Il expose aussi le rôle de médiateur et de symbole d'une Église qui est resté elle-même quand d'autres ont divergé en cherchant le pouvoir
    et la domination. En ce sens, il affirme la constance d'une Église qui se rapproche du message original de l'Église primitive.

    \subsection*{Un essai romancé qui témoigne de la fierté d'un peuple délaissé}

    Cet essai, dans les mains d'arménophiles, est un bijou.
    Il résume, ordonne et expose les principauxs faits historiques de l'Arménie et de son Église.\\
    Cependant, même si ce livre est une parfaite introduction à l'étude de l'incroyable culture arménienne, il reste assez court.\\
    De plus, l'œuvre datant de 1979, quand l'Armenie était sous le joug soviétique, 
    laisse une interrogation sur l'honneteté intelectuelle de celle-ci, même si l'auteur est français.
    
    D'après \Citeauthor{livre:geopol}, cette période est une période spéciale entre la diaspora et le peuple resté en Arménie,
    en effet, une tension règne entre \enquote{les deux pans du monde arménien}\cite{livre:geopol}.
    Avec une émérgence d'un pseudo-nationalisme de la diaspora. Ce phénomène se ressent dans l'essai,
    en effet, l'auteur exacerbe l'importance de l'Arménie dans certains contextes qui peut laisser un doute sur la totale véracité des propos.
    \bigbreak

    Ceci dit, en comprenant le contexte dans lequel cet essai a été écrit, 
    on comprend l'amour que l'auteur pourte pour sa nation d'origine et l'Église qu'elle représente.
    Voici quelques exemples :
    
    \enquote{Quand on se penche sur l'histoire de l'Arménie, on est surpris qu'un si petit peuple, disséminé à travers les vallées caucasiennes,
    à la croisée des civilisations, ait réussi non seulement à résister à des forces infiniment supérieures aux siennes,
    à des volontés farouchement décidées à l'écraser, à des ennemis extrêmement sauvages et agressifs, 
    mais encore à former une nation avec identité de langue, de mœurs, 
    de religion et une civilisation tellement importante et originale}\cite{livre:principale}
    
    \enquote{Cette Église si peu importante par le nombre de ses fidèles\footnote{\enquote{On en compte environ 6 millions}\cite{livre:principale}}
    a marqué l'histoire de l'Eglise Primitive (seule et véritable Eglise Universelle) par la qualité de ses représentants,
    sa fermeté dans la défense des dogmes et son libéralisme dans l'élaboration des doctrines.}\cite{livre:principale}

    \enquote{Si la Civilisation arménienne depuis près de deux mille ans ne peut se concevoir hors du Christianisme, 
    si la Foi pour laquelle les Arméniens ont sacrifié leur confort, leur liberté et leur vie,
    fait partie d'eux-mêmes ils ont aussi incarné un apostolat et un zèle missionnaire peu commun pour une si petite nation.}\cite{livre:principale}

    \enquote{Lorsque l'on considère la longue suite de massacres et de dévastations qui ont fait disparaître de la carte des cités innombrables,
    dépeuplé de vastes provinces, provoqué les migrations de populations entières,
    a refusé l'assimilation ou la conversion qui auraient mis fin à ses malheurs. 
    À cette admiration se mêle de la reconnaissance car, si la Nation Arménienne avait disparu,
    si elle s'était fondue dans les nations voisines, 
    il manquerait à l'édifice de la civilisation une pierre... peut-être même une coupole}\cite{livre:principale,citation:alem}
    \footnote{Ici, il y a une citation dans le livre, je cite une citation.}

    \subsection*{L'Arménie, un petit pays, avec une grande histoire}

    Pour conclure, cet essai est un effort d'une reconnaissance de la culture arménienne, en particulier de son Église.
    On y retrouve une exposition historique (parfois romancée) de l'origine de la civilisation arménienne ainsi que la chrétienté.
    L'auteur nous fait découvrir, avec engouement, les travers et subtilités de l'Église apostolique arménienne.
    Un témoin d'une tradition perdue par la majorité des pratiquants.\\
    On comprend les défis que représente d'être un petit pays représentant la minorité religieuse chrétienne au sien d'un cercle majoritairement musulman. 
    De plus, l'auteur nous laisse la mission de préservation et de médiation que peut représenter une Église qui ne demande 
    qu'à la réconciliation entre les croyants. Cet essai ce conclu par le récit d'une vie, celle d'un Catholicos\footnote{Équivalent au Pape Catholique}  
    apprend l'existence de l'imprimerie et en fait sa mission d'imprimer en arménien, il se heurta au Pape ne voyait pas son confrère comme égale. 
    C'est un message fort qui résume l'histoire de l'Église arménienne.
    \pagebreak
    \printbibliography
\end{document}